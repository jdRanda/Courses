\documentclass[a4paper, 12pt, twoside]{article}
\usepackage[left = 3cm, right = 3cm]{geometry}
\usepackage[english]{babel}
\usepackage[utf8]{inputenc}
\usepackage{amssymb}
\usepackage{amsmath}
\usepackage{amsthm}
\usepackage{multicol}
% thanks tyler

\author{jdRanda}
\title{Set Theory}

\begin{document}
\maketitle
\newpage
\section{Introduction}
\subsubsection*{Axiom of Extensionality (for sets)}
For two sets $a,b$, we say $a=b$ iff:
$$ \forall x,[x\in a\Rightarrow x\in b]$$
\subsubsection*{Axiom of Extensionality (for classes)}
For two classes $A,B$, we say $A=B$ iff:
$$ \forall x,[x\in A\Rightarrow x\in B]$$
\subsubsection*{Axiom of Pair Set}
For any sets $x,y$ there is a set $z=\{x,y\}$ with elements just $x$ and $y$. $z$ is called the \emph{(unordered) pair set of $x,y$}.
\textit{NB: If $x=y$ then we have $\{x,y\}=\{x,x\}=\{x\}$}
\subsection*{Definition 1.1}
Let $\mathcal{P}(x)$ denote the class $\{y|y\subseteq x\}$, called the \emph{Power set of $x$}.
\subsection*{Definition 1.2}
The \emph{Empty Set}, denoted $\emptyset$, is the unique set with no elements.
\begin{itemize}
    \item We can define $\emptyset$ as $\{x|x\neq x\}$.
    \item For any set/class $A$, we have $\emptyset \subset A$.
\end{itemize}

\newpage
\section{Classes}
\subsection*{Theorem 1.4}
The collection $R=\{x|x\notin x\}$ does not define a set.\\
\begin{proof}
    Suppose $R$ was a set $z$. Assume $z\in R$, then by the definition of $R$, $z \notin z$. However if $z \notin R$ then we should have $z \in z$. This is a contradiction.
\end{proof}
\textit{NB: This is an example of a class that is not a set. Any class which is not, or cannot be, a set is called a \emph{Proper Class}.}
\subsubsection*{Axiom of Subsets}
Let $\Phi(x)$ be a definite, welldefined property. Let $x$ be a set. Then
$$ \{y\in x|\Phi(y)\}\quad\text{is a set.}$$
\subsection*{Corollary 1.5}
Let $V$ denote the class of all sets. Then $V$ is a proper class.
\begin{proof}
    If $V$ were a set then we should have, $R=\{y\in V|y\notin y\}$ is a set by the Axiom of Subsets, however we  have just shown R is not a set.
\end{proof}
\subsection*{Definition 1.6}
For any set $Z$ there is a class, $\cup Z$, which consists of the members of members of $Z$.
$$\cup Z=\{x|\exists t (x\in t\in Z)\}$$
\subsubsection*{Axiom of Unions}
For any set $Z$, $\cup Z$ is a set.
\subsection*{Definition 1.8}
For any non-empty set $Z$, there is another set, $\cap Z$, which consists of the members of all members of $Z$.
$$\cap Z=\{x|\forall t\in Z(x\in t)\}$$
or using index sets we write
$$x\in \cap_{j\in I}A_{j}\iff (\forall j\in I)(x \in A_{j}) $$

\newpage
\section{Relations and Functions}
For two sets $X,Y$, there are relations $R$ that hold between some elements of $X$ and of $Y$, denoted $xRy$. The types of relation are:
\begin{itemize}
    \item Reflexive:  $x\in X \Rightarrow xRx$
    \item Irreflexive:  $x\in X \Rightarrow \neg (xRx)$
    \item Symmetric:  $(x,y\in X \wedge xRy) \Rightarrow yRx$
    \item Antisymmetric: $(x,y\in X \wedge xRy  \wedge yRx)\Rightarrow x=y$
    \item Connected:  $(x,y\in X) \Rightarrow (x=y\vee xRy\vee yRx)$
    \item Transiative:  $(x,y,z\in X \wedge xRy \wedge yRz) \Rightarrow xRz$
\end{itemize}
\textit{NB: recall that an \emph{equivalence relation} is that R should satify symmetry, reflixivity and transiativity.}
\subsection*{Definition 1.10}
A relation $\prec$ on a set $X$ is a (strict) partial ordering if it is irreflexitve and transitive. I.e.
\begin{itemize}
    \item[i)]$x\in X\Rightarrow \neg (x\prec x)$.
    \item[ii)] $(x,y,z\in X \wedge x \prec y \wedge y \prec z)\Rightarrow (x\prec z)$
\end{itemize}


\subsection*{Definition 1.11}
\begin{itemize}
    \item[i)] If $\prec$ is a partial ordering of a set $X$, and $\emptyset \neq Y \subseteq X$, then $z\in X$ is a lower bound for $Y$ in $X$ if:
    $$\forall Y(y\in Y\Rightarrow z \preceq y) $$
    \item[ii)] $z\in X$ is an infimum or greatest lower bound (glb) for $Y$ if it is a lower bound for Y and if $z'$ is a lower bound for $Y$ then $z'\preceq z$.
    \item[iii)] The concepts of upper bound and supremum (least upper bound (lub)) are defined analogously.
\end{itemize}
\subsection*{Definition 1.12}
\begin{itemize}
    \item[i)] We say $f:(X,\prec_{1})\to(Y,\prec_{2})$ is an order preserving map of the partial orders $(X,\prec_{1}),(Y,\prec_{2})$ iff:
    $$\forall x,z\in X (x,\prec_{1}z \Rightarrow f(x),\prec_{2}f(z)) $$
    \item[ii)] Orderings $(X,\prec_{1}),(Y,\prec_{2})$ are (order) isomorphic, written $(X,\prec_{1})\cong(Y,\prec_{2})$, if there is an order preserving map between them which is also a bijection.
    \item[iii)] There are completely analogous definitions between nonstrics orders $\preceq_{1}, \preceq_{2}$.
\end{itemize}
\subsection*{Theorem 1.13\textit{(Representation Theorem for partially ordered sets)}}
If $\prec$ partially orders $X$, then there is a set $Y$ of subsets of $X$ which is such that $(X,\preceq)$ is order isomorphic to $(Y,\subseteq)$.
\begin{proof}
    Given any $x\in X$, let $X^{x} = \{z\in X|z\preceq x\}$. Notice if $x\neq y$ then $X^{x}\neq X^{y}$. So the assignemnt of $x$ to $X^{x}$ is 1:1. Let $Y = \{X^{x}|x\in X\}$. Then we have
    $$x\preceq y \iff X^{x}\subseteq X^{y} $$
    Coonsequently, setting $f(x)=X^{x}$ we have an order isomorphism.
\end{proof}
\textit{NB: Often we deal with roderings where every element is  comparable with every other, known as \emph{strong connectivity} and we call the rdering \emph{total}.}
\subsection*{Definition 1.14}
A relation $\prec$ on $X$ is a string total ordering if it is a partial ordering which is connected:
$$\forall x,y (x,y\in X\Rightarrow (x=y\vee x\prec y\vee y\prec x)) $$
\textit{NB: For $\preceq$ we call the ordering non-strict.}
\subsection*{Definition 1.15}
\begin{itemize}
    \item[i)]$(A,\prec)$ is a wellordering if it is a string total orderings and for any subset $Y\subseteq A$, $Y\neq \emptyset\Rightarrow Y$ has a $\prec$-least element. We write $(A,\prec)\in WO$
    \item[ii)] A partial ordering $R$ on a set $A$, $(A,R)$ is a wellfounded relation if for any subset $Y\subseteq A$, $Y\neq \emptyset\Rightarrow Y$ has an $R$-minimal element.
\end{itemize}
\subsection*{Lemma 1.16}
A strict total ordering $(A,\prec)$ is a wellordering iff any non-empty \emph{end segment} $C\subseteq A$, has a $\prec$-least element.
\textit{We say $C\subseteq A$ is an end segment of the strict total order $(A,\prec)$, if whenever $a\in C$ and $a\prec b$, then $b\in C.$}
\begin{proof}

\end{proof}
\subsection*{Definition 1.17 (Kuratowski)}
Let $x,y$ sets. The ordered pair set of $x$ and $y$ is the set
$$\langle x,y \rangle=\{\{x\},\{x,y\}\}$$
\subsection*{Lemma 1.18 (Uniqueness theorem for ordered pairs)}
$$\langle x,y \rangle = \langle u,v \rangle \iff x=u\wedge y=v$$
\begin{proof}
    $(\Leftarrow)$ is trivial. So Suppose $\langle x,y \rangle = \langle u,v\rangle$.\\
    \begin{itemize}
        \item[Case 1] $x=y$. Then $\langle x,y \rangle = \langle x,x\rangle= \{\{x\},\{x,x\}\}=\{x\{x\},\{x\}\}=\{\{x\}\}$. If this equals $\langle  x,v\rangle$ then we must have $u=v$. So $\langle u,v \rangle=\{\{u\}\}=\{\{x\}\}$. Hence by Extensionality $\{u\}=\{x\}$, and so again by Extensionality $u=x=y=v$.
        \item[Case 2] $x\neq y$. Then $\langle x,y \rangle , \langle u,v\rangle$ have the same two elements, ($u\neq v$). Hence one of these elements has one member and the other two, so we cannot have $\{x\}=\{u,v\}$. So $\{x\}=\{u\}$  and $x=u$. But that means $\{x,y\}=\{u,y\}=\{u,v\}$. So of these last two sets, if they are the same then $y=v$.
    \end{itemize}
\end{proof}
\subsection*{Definition 1.20}
We define ordered k-tuple by induction: $\langle x_{1},x_{2} \rangle$ has been defined; if $\langle x_{1},x_{2}.\dots,x_{k}\rangle$ has been defined then $\langle x_{1},\dots,x_{k},x_{k+1}\rangle=\langle \langle x_{1},\dots,x_{k}\rangle, x_{k+1}\rangle$.
\subsection*{Definition 1.21}
\begin{itemize}
    \item[i)] Let $A,B$ be sets. $A\times B=\{\langle x,y\rangle|x\in A \wedge y \in B\}$. If $A=B$ this is written as $A^{2}$.
    \item[ii)] If $A_{1},\dots,A_{k+1}$ sets we define $A_{1}\times\dots\times A_{k+1}=(A_{1}\times\dots\times A_{k})\times A_{k+1}=
    \{\langle\dots \langle\langle\langle x_{1},x_{2}\rangle,x_{3} \rangle,\dots,x_{k}\rangle,x_{k+1}\rangle|\forall i(1\leq i\leq k+1\Rightarrow x_{i}\in A_{i}). \}$
    \item[iii)] In general $A\times B \neq B\times A$, and further $\times$ operation is not associative.
\end{itemize}
\subsection*{Definition 1.22}
\begin{itemize}
    \item[i)] A (binary) relation $R$ is a class of ordered pairs. R is thus any subset of some $A\times B$.
    \item[ii)] We write $R^{-1}=\{\langle y,x\rangle|\langle x,y\rangle\in R$.
\end{itemize}
\subsection*{Definition 1.24}
If R is a relation, then
$$dom(R)=\{x|\exists y \langle x,y\rangle \in R\},ran(R)=\{y|\exists y \langle x,y\rangle \in R\}$$
The field of a relation R, $ Field(R)=dom(R)\cup ran(R)$.
\subsection*{Definition 1.25}
\begin{itemize}
    \item[i)] A relation $F$ is a function $("Func(F)")$ if $\forall x \in dom(F)$ there is a unique $y$ s.t $\langle x,y\rangle \in F$.
    \item[ii)] If F is a function then $F$ is $(1-1)$ iff $\forall x,x'(\langle x,y\rangle\in F\wedge \langle x',y\rangle\in F \Rightarrow x=x')$
\end{itemize}
\subsection*{Definition 1.27}
If $X,Y$ sets, then $^{X}Y=\{F|F:X\to Y$.
\subsection*{Definition 1.28 (Indexed Cartesia Product)}
Let $I$ be a set, and for each $i\in I$ let $A_{i}=\emptyset$ be a set; then
$$\prod_{i\in I}A_{i}=\{f|Func(f),dom(f)=I\wedge\forall i\in I(f(i)\in A_{i})\}$$
This allows us to take cartesian product indexed by any set, not just some finite $n$.
\textit{NB: Our function $f$ can be seen as a `choice' function that choose some $f(i)\in A_{i}$ for each $i$.}
\subsection*{Definition 1.30}
A set $x$ is \emph{transiative}, $Trans(x)$, iff $\forall y\in x(y\subseteq x)$.
\textit{NB:  We also equivalently abbreviate $Trans(x)=\cup x\subseteq x$}
\subsection*{Definition 1.32 (The successor function)}
Let $x$ a set. Then $S(x)=x\cup\{x\}$.
\subsection*{Examples 1.33}
If $x$ is transitive then so too is $S(x)$.
\begin{proof}
    Assume $x$ is transiative. Let $y\in S(x)$. If $y\in x$ then $y\subseteq x$ as x is transitive. Then $y\subseteq S(x)$. Else $y=x$, trivially $y\subseteq s(x)$, Hence transitive.
\end{proof}
For $X$ a class of transitive sets. Then $\cup X$ is transitive.
\begin{proof}
    Let $y\in X$. Want to show $y\subseteq \cup X$. So let $z\in y$, as $y\in \cup X$ theres some $t \in X, Trans(t)$, with $y\in t$. Then $z\in y\subseteq t$. So $z\in t\in X$. So $z in \cup X$. Hence $y\subseteq \cup X$.
\end{proof}

\subsection*{Lemma 1.33}
$ Trans(x)\iff \cup S(x)=x$
\begin{proof}
    First note that $\cup S(x)=\cup(x\cup\{x\})=(\cup x) \cup (\cup\{x\})=(\cup x) \cup x$. For $(\rightarrow)$ , assume $Trans(x)$; then $\cup x \subseteq x$. Hence by the above $\cup S(x)\subseteq x$. Hence $\cup S(x)= x$.\\
    For $(\leftarrow)$, assume $\cup S(x)=x$. We have from above $\cup x \subseteq (\cup x)\cup x = x$ by assumption. Hence transitive.
\end{proof}
\subsection*{Definition 1.34 (Transiative Closure)}
We define by recursion on $n$:
$$\cup^{0}x=x;\cup^{n+1}x=\cup(\cup^{n}x); TC(x)=\cup\{\cup^{n}x|n\in \mathbb{N}\} $$
\subsection*{Lemma 1.35}
For any set $x$
\begin{itemize}
    \item[i)] $x\subseteq TC(x)$, $Trans(TC(x))$.
    \item[ii)] $Trans(t)\wedge x\subseteq t$ then $TC(x)\subseteq t$. Hence $TC(x)$ is the smallest transitive set containing x.
    \item[iii)] $Trans(x)\iff TC(x)=x$.
\end{itemize}
\begin{proof}
    \begin{itemize}
        \item[i)] trivial.
        \item[ii)] $x\subseteq t$ then $\cup^{0}\subseteq t$. By induction on k, assume $\cup^{k}\subseteq t$. Now use the fact $A\subseteq B \wedge Trans(B)\Rightarrow \cup A\subseteq B$ to deduce $\cup^{k+1}\subseteq t$. So it follows $TC(x)\subset t$. But $t$ was arbitrary.
        \item[iii)] $x\subseteq TC(x)$, if $Trans(x)$ then substitude $x$ for $t$ in the above, concluding $TC(x)\subseteq x$.
    \end{itemize}
\end{proof}

\section{Number Systems}
\subsection*{Definition 2.1}
A set $Y$ is called inductive if $\emptyset \in Y$ and $\forall x\in Y (S(x)\in Y)$.
\textit{Axiom of Infinity: There exists an inductive set, $\exists Y(\emptyset\in Y \wedge \forall x\in Y (S(x)\in Y))$.}
\subsection*{Definition 2.2}
\begin{itemize}
    \item[i)] $x$ is a natural number if $\forall Y[Y\text{ is an inductive set}\to x\in Y]$.
    \item[ii)] $\omega$ is the class of natural numbers.
\end{itemize}
\textit{NB: $\omega=\cap\{Y|Y\text{ an inductive set}\}$}
\subsection*{Proposition 2.3}
$\omega$ is a set.
\begin{proof}
    Let $z$ be any inductive set. By the Axiom of subsets: there is a set $N$ so that:
    $$N=\{x\in z|\forall Y[Y \text{ an inductive set}\to x\in Y] $$
\end{proof}
\subsection*{Proposition 2.4}
\begin{itemize}
    \item[i)] $\omega$ is an inductive set.
    \item[ii)] It is the smallest inductive set.
\end{itemize}
\begin{proof}
    We have proven $\omega$ is a set. To show inductivity, not by definition $\emptyset$ is in any inductive set $Y$ so $\emptyset\in\omega$. Moreover, if $x\in\omega$, then for any inductive set $Y$, we have both $x,S(x)\in Y$. Hence $S(x)\in\omega$. So $\omega$ closed under the $S$ function. (ii) then follows.
\end{proof}
\subsection*{Theorem 2.5 (Principle of Mathematical Induction)}
Suppose $\Phi$ is a welldefined definite propery of sets. Then
$$[\Phi(0)\wedge\forall x\in\omega(\Phi(x)\to\Phi(S(x)))\to\forall x\in \omega\quad\Phi(x)] $$
\begin{proof}
    Assume the entecedent here, then it suffices to show that the set of $x\in\omega$ for which $\Phi(x)$ holds is inductive. Let $Y=\{x\in\omega|\Phi(x)\}$. However the antecedent then says $0\in Y$; and moreover if $x\in Y$ then $S(x)\in Y$. That $Y$ is inductive is then simply the antededent assumption. Hence $\omega\subseteq Y$. And so $\omega = Y$.
\end{proof}
\subsection*{Proposition 2.6}
Every natural number $y$ is either $0$ or is $S(x)$ for some natural number $x$.
\begin{proof}
    Let $Z=\{y\in\omega|y=0\vee\exists x\in\omega(S(x)=y)\}$. Then $0\in Z$ and if $u\in Z$, then $u\in \omega$. Hence $S(u)\in\omega$ as $\omega$ inductive. Hence $S(u)\in Z$, so $Z$ is inductive and thus $\omega$.
\end{proof}
\subsection*{Exercise 2.1}
Every natural number is transitive
\begin{proof}
    Wts $Z=\{x\in\omega|Trans(x)\}$ is inductive.
\end{proof}
\subsection*{Lemma 2.7}
$\omega$ is transiative.
\begin{proof}
    Let $X=\{n\in\omega|n\subseteq\omega\}$. If $X$ were inductive, then $X\subseteq\omega\subseteq X$, and then $Trans(\omega)$. Trivially $\emptyset\in X$. Assume $n\in X$, then $n\subseteq \omega$ and $\{n\}\subseteq\omega$. Hence $S(n)\in X$. So $X$ is inductive.
\end{proof}
\subsection*{Definition 2.10}
For $m,n\in\omega$ set $m<n\iff m\in n$. Set $m\leq n \iff m=n\vee m<n$
\subsection*{Lemma 2.11}
\begin{itemize}
    \item[i)] $<, \leq$ are transitive.
    \item[ii)] $\forall n\in\omega \forall m(m<n\iff S(m)<S(n))$.
    \item[iii)] $\forall m\in\omega(m\not<m)$.
\end{itemize}
\begin{proof}
    \begin{itemize}
        \item[i)] That $<$ is transitive comes from the fact our natural numbers are proven to be transitive sets: $n\in m\in k\Rightarrow n\in k$. The same follows for $\leq$.

        \item[ii)]
        \item[iii)]
    \end{itemize}
\end{proof}

\end{document}
