\documentclass[a4paper, 12pt, twoside]{article}
\usepackage[left = 3cm, right = 3cm]{geometry}
\usepackage[english]{babel}
\usepackage[utf8]{inputenc}
\usepackage{amssymb}
\usepackage{amsmath}
\usepackage{amsthm}
\usepackage{multicol}

\author{jdRanda}
\title{Lecture 1}

\begin{document}
    \section{Intrduction}
        \subsection{Complex Numbers}
            Defined as $ \mathbb{C} = \{ (x,y):x,y \in \mathbb{R} \} $
            subject to conditions,  for $(x_{1}, y_{1}), (x_{2}, y_{2}) \in \mathbb{C}$
            \begin{itemize}
                \item Addition ($+$):\\
                $ (x_{1}, y_{1}) + (x_{2}, y_{2}) = (x_{1}+x_{2}, y_{1}+y_{2})$
                \item Multiplication ($\cdot$):\\ $ (x_{1}, y_{1}) + (x_{2}, y_{2}) = (x_{1}x_{2}-y_{1}y_{2}, x_{2}y_{1}+x_{1}y_{2}) $
            \end{itemize}
            $(\mathbb{C},+), (\mathbb{C},\cdot)  $ are albelian groups, with units $(0,0)$ and $(1,0)$ respectivly
            \subsubsection{Lemma}
                $(\mathbb{C},+,\cdot) $ is a field with multiplicative inverse
                $$z \in \mathbb{C} \text{\textbackslash}  \{0,0\}, \quad z^{-1}=(\frac{x}{x^{2}+y^{2}},\frac{-y}{x^{2}+y^{2}})$$\\
                such that $z*z^{-1}=(1,0)$\\\\

            \noindent We will define as follows:
            $$ 1:=(1,0),\quad i:=(0,1),\quad 0:=(0,0) $$
            Allowing us to write complex numbers as


        \newpage
        \subsection{Sequences, Series and Convergence}
        The space $(\mathbb{C},|.|)$ is a \emph{metric space} which can be identified with $\mathbb{R}^{2}$ with the euclidean distance.
            \subsubsection{Definition}
            A sequence $\{z_{n}\}_{n\in \mathbb{N}}$ converges, $\lim_{n\to \infty}z_{n}=w$ iff
            $$\lim_{n\to \infty}|z_{n}-w|=0 $$
            Since $\mathbb{R}^{2}$ is complete, so is $\mathbb{C}$. Namely every Cauchy sequence in $\mathbb{C}$ has a limit in $\mathbb{C}$.
            \subsubsection{Proposition}
                A sequence $\{z_{n}\}_{n\in \mathbb{N}}$ converges iff
                $$ \forall\epsilon>0,\exists N>0, \text{s.t. } n,m>N \Rightarrow |z_{n}-z_{m}|<\epsilon$$\\
                \noindent
                A series $ \sum_{n=0}^{\infty}z_{n}$ converges iff the sequence of partial sums $(s_{n})$ converges
                $$\text{where } s_{n}=\sum_{j=0}^{n}z_{j}$$
                \noindent
                The series of real numbers $ \sum_{n=0}^{\infty}b_{n}, \text{for } b_{n}\geq 0$ converges if $\lim\sup_{n\to\infty}|b_{n}|^{1/n}<1$ , and diverges to $+\infty$ if $\lim\sup_{n\to\infty}|b_{n}|^{1/n}>1$

            \subsubsection{Theorem}
                If a series $ \sum_{n=0}^{\infty}z_{n}$ converges absolutely, then it too converges.\\
                \underline{proof}\\
                Let $T_{n}=\sum_{j=0}^{n}|z_{j}|,\quad S_{n}=\sum_{j=0}^{n}z_{j}$. Since the series converges absolutley, we know for some $N\in \mathbb{N}$
                $$|T_{n}-T_{m}|=|a_{n}|+\dots+|a_{m+1}|<\epsilon \text{ for }n>m>N, \epsilon>0$$
                and by the triangle inequality we can show
                \begin{align*}
                    |S_{n}-S_{m}|&=|a_{n}+\dots+a_{m+1}|\\
                    &\leq |a_{n}|+\dots+|a_{m+1}| \\
                    &=|T_{n}-T_{m}|< \epsilon
                \end{align*}
                 Hence the sequence $(S_{n})$ is cauchy and must converge\qed
        \newpage
        \section{Holomorphic Functions}
            \subsection{Differentiation of Complex functions}
                \subsubsection{Definition}
                    A subset $G\subset \mathbb{C}$ is called a \emph{domain} if it is open and connected.
                \subsubsection{Definition}
                    A function $f:G \to  \mathbb{C}$ has a limit $c$ at a point $z_{0}\in G$ if
                    $$\forall\epsilon>0,\exists\delta>0\text{ s.t. } |z-z_{0}|<\delta \Rightarrow |f(z)-c|<\epsilon$$
                \subsubsection{Definition}
                    The function $f$ is continuous at a point $z_{0}\in G$ if the limit of $f$ at $z_{0}$ exists and
                    $$ \lim_{z\to z_{0}}f(z)=f(z_{0})$$
                    If $f$ is continuous at every point $z\in G$ then f is continuous in $G$.\\\\
                    We use the notation
                    $$z = x+iy,\quad f(z)=u(x,y)+iv(x,y) $$
                    Where $u(x,y),v(x,y)$ are functions from $\mathbb{R}^{2}\to \mathbb{R}$. From this we can write
                    $$ Re(f)=u,\quad Im(f)=v$$
                    So we can say $f$ is continious in $G \iff u,v$ are continious in $G$.
                \subsubsection{Definition}
                    Let $G$ be a domain in $\mathbb{C}$ and $f:G \to  \mathbb{C}$ a complex function on $G$. The function $f$ is (complex) differentiable at a point $z_{0}\in G$ iff the limit
                    $$ f'(z_{0})=\lim_{h\to0}\frac{f(z_{0}+h)-f(z_{0})}{h}\quad[= \lim_{z\to z_{0}}\frac{f(z)-f(z_{0})}{z-z_{0}}]$$
                    exists and is finite.
                    The limit is independant from the direction in the complex plane in which $h$ tends to zero.
            \subsection{Basic Properties of Complex Differentiation}
                Complex differentiability shares several properties with real differentiability. Those being it is linear and obeys the product rule, quotient rule, and chain rule.\\\\
                Let $G\subset \mathbb{C}$ be an open set and $f:G \to  \mathbb{C}, g:G \to  \mathbb{C}$ complex functions on $G$, with $z_{0}\in G$.
                \begin{itemize}
                    \item[i)]  Suppose $f,g$ are differentiable at $z_{0}$ then  $f+g,af,fg (a\in \mathbb{C})$ are also differentiable at said point and
                    \noindent
                    $$(f+g)'(z_{0})=f'(z_{0})+g'(z_{0}), (af)'(z_{0})=af'(z_{0}) $$
                    $$(fg)'(z_{0})= f'(z_{0})g(z_{0})+g'(z_{0})f'(z_{0}) $$
                    \item[ii)]  Suppose $f,g$ are differentiable at $z_{0}$ and $(g)'(z_{0})\neq 0$. Then $f/g$ is differentiable at the point and
                    $$(\frac{f}{g})'(z_{0})=\frac{f'(z_{0})g(z_{0})-g'(z_{0})f(z_{0})}{g^{2}(z_{0})} $$
                    \item[iii)] Suppose $f$ is differentiable at $z_{0}$, then $f$ is continuous at $z_{0}$.\\
                    \underline{proof}\\
                    As $f$ is differentiable at $z_{0}$ then we know $\lim_{z\to z_{0}}\frac{f(z)-f(z_{0})}{z-z_{0}}=f'(z_{0}) $ and we wts $ \lim_{z\to z_{0}}f(z)=f(z_{0})$. So
                    \begin{align*}
                         \lim_{z\to z_{0}}(f(z)-f(z_{0})) &=  \lim_{z\to z_{0}}(z-z_{0})\frac{(f(z)-f(z_{0}))}{z-z_{0}}\\
                         &=[\lim_{z\to z_{0}}(z-z_{0})][\lim_{z\to z_{0}}\frac{(f(z)-f(z_{0}))}{z-z_{0}}]\\
                         &= 0\cdot f'(z_{0}) =0
                    \end{align*}\qed\newpage
                    \item[iv)] Let $B\subset \mathbb{C}$ also be an open set. Suppose $f$ is differentiable at $z_{0},f(G)\subset B$ and $g:B\to \mathbb{C}$ is differentiable at $f(z_{0})\in B$. Then we can say the composition $g\circ f:G\to \mathbb{C}$ is differentiable at $z_{0}$ and
                    $$(g(f))'(z_{0})=g'(f(z_{0}))f'(z_{0}) $$
                    \underline{proof}\\
                    We know $g,f$ are differentiable at $z_{0}, f(z_{0})$ respectivly. Thus we can write
                    $$\lim_{h\to0}\frac{f(z_{0}+h)-f(z_{0})}{h}=f'(z_{0}),\quad \lim_{h\to0}\frac{g(f(z_{0})+h))-g(f(z_{0}))}{h} =g'(f(z_{0}))$$
                    Now lets define
                    $$v:= \frac{f(z_{0}+h)-f(z_{0})}{h}-f'(z_{0}), \quad
                    u:=\frac{g(f(z_{0})+h))-g(f(z_{0}))}{h} -g'(f(z_{0}))$$
                    This allows us to write
                    $$f(z_{0}+h)=f(z_{0})+[f'(z_{0})+v]h,\quad g(f(z_{0})+k)=g(f(z_{0}))+[g'(f(z_{0}))+u]k $$
                    So we can say
                    $$g(f(z_{0}+h))= g(f(z_{0})+[f'(z_{0})+v]h)$$
                    and then
                    \begin{align*}
                        \frac{g(f(z_{0}+h))-g(f(z_{0}))}{h} &= \frac{g(f(z_{0}))+[g'(f(z_{0}))+u]\cdot[f'(z_{0})+v]\cdot h- g(f(z_{0}))}{h}\\
                        &=[g'(f(z_{0}))+u]\cdot[f'(z_{0})+v]\\
                        \lim_{h\to0}\frac{g(f(z_{0}+h))-g(f(z_{0}))}{h} &=
                        \lim_{h\to0}[g'(f(z_{0}))+u]\cdot[f'(z_{0})+v]\\
                        &=[\lim_{h\to0}g'(f(z_{0}))+\lim_{h\to0}u]\cdot[\lim_{h\to0}f'(z_{0})+\lim_{h\to0}v]\\
                        &=g'(f(z_{0}))f'(z_{0})
                    \end{align*}
                    \textit{as both $u,v\to 0 $ as $h\to 0$.}\\
                    Hence the limit exists and is finite so we can say $g\circ f$ is differentiable at $z_{0}$ and
                    $$(g(f))'(z_{0})=g'(f(z_{0}))f'(z_{0}) $$\qed

                \end{itemize}
                \subsubsection{Definition}
                    A function $f:G \to  \mathbb{C}$ defined on a domain $G$ is called \emph{holomorphic} in $G$ if it has a complex derivative at all points in $G$.
            \newpage
            \subsection{Derivative as a linear approximation}
                \subsubsection{Definition}
                    Let $w(z)$ be a complex function in a neighbourhood of $z=0$. The function $w(z)=o(z)$ if
                    $$\lim_{z\to0}\frac{w(z)}{z}=0$$
                    $w(z)=O(z)$ if
                    $$\lim_{z\to0}\frac{w(z)}{z}=c,\quad c\neq 0$$
                \subsubsection{Lemma}
                    Let $G$ be a domain in $\mathbb{C}$ and $f:G \to  \mathbb{C}$ a compelx function on $G$. $f$ is complex differentiable at $z_{0}\in G$ iff there exists a constant $A\in \mathbb{C}$ s.t.
                    $$ f(z_{0}+h)-f(z_{0})=Ah+o(h),\quad \forall h\in G_{z_{0}}$$
                    \textit{where $G_{z_{0}}$ is a neighbourhood of $z_{0}$ in $G$ and $A=f'(z_{0})$.}\\
                    \underline{Proof}\\
                    Suppose $f$ differentiable at $z_{0}$. Let $A=f'(z_{0})$. Define $w:G_{z_{0}} \to  \mathbb{C};h\to f(z_{0}+h)-f(z_{0})-Ah$. Then
                    $$\lim_{h\to0}\frac{w(h)}{h}=\lim_{h\to0}\frac{f(z_{0}+h)-f(z_{0})-f'(z_{0})h}{h}=f'(z_{0})-f'(z_{0})=0$$
                    Hence $w(h)=o(h)$ is satified.\\
                    Now supppose that $A\in \mathbb{C}, w:G_{z_{0}} \to  \mathbb{C};h\to f(z_{0}+h)-f(z_{0})-Ah=o(h)$. Then
                    $$ \lim_{h\to0}\frac{f(z_{0}+h)-f(z_{0})}{h}=\lim_{h\to0}\frac{Ah+w(h)}{h}= A+\lim_{h\to0}\frac{w(h)}{h}=A$$
                    It follows that $\lim_{h\to0}\frac{f(z_{0}+h)-f(z_{0})}{h}$ exists and $f'(z_{0})=A$.\qed
            \newpage
            \subsection{Real versus Complex Differentiation}
                We consider arbitrary maps of the form
                $$F:\mathbb{R}^{2}\to \mathbb{R}^{2};(x,y)\to(u(x,y),v(x,y)) $$
                \subsubsection{Theorem}
                    The function $F$ is real differentiable at $(x_{0},y_{0})$ if there exists a linear map $A:\mathbb{R}^{2} \to  \mathbb{R}^{2}$ s.t.
                    $$F(x,y)-F(x_{0},y_{0})=(x-x_{0},y-y_{0})A+o(\sqrt{(x-x_{0})^{2}+(y-y_{0})^{2}})$$
                    \textit{The matrix $A$ is the transpose of the Jacobian of $F$ calculated in $(x_{0},y_{0})$, namely}
                    $$A^{t}=JF(x_{0},y_{0})=
                    \begin{pmatrix}
                            u_{x}(x_{0},y_{0}) & u_{y}(x_{0},y_{0})\\
                            v_{x}(x_{0},y_{0}) & v_{y}(x_{0},y_{0})
                    \end{pmatrix}$$
                    and
                    $$\lim_{x\to x_{0},y\to y_{o}}\frac{o(\sqrt{(x-x_{0})^{2}+(y-y_{0})^{2}})}{\sqrt{(x-x_{0})^{2}+(y-y_{0})^{2}}}=0 $$

            \newpage
            \subsection{Caughy-Riemann Equations}
                \subsubsection{Cauchy-Riemann Theorem}
                    Suppose $f:G \to  \mathbb{C};z\to u(x,y)+iv(x,y)$ is complex differentiable in $z_{0}\in G$. Then $u,(x,y)$ and $v(x,y)$ satisfy the following equations:
                    $$ $$







\end{document}
