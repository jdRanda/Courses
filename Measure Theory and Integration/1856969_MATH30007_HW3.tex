\documentclass[a4paper, 12pt, twoside]{article}
\usepackage[left = 3cm, right = 3cm]{geometry}
\usepackage[english]{babel}
\usepackage[utf8]{inputenc}
\usepackage{amssymb}
\usepackage{amsmath}
\usepackage{amsthm}
\usepackage{multicol}

\author{Joseph}

\begin{document}

Joseph Dowling.
\subsection*{Exercise 1}
\subsection*{Exercise 4}
Let $\epsilon>0$. We know by definition $f=f^{+}-f^{-}$, and $\int fd\mu=\sup\{\int sd\mu|o\leq s\leq f,s \text{simple and measurable}\}$. So we have that
$$\exists \gamma^{+},\gamma^{-} \text{ simple,measurable s.t. } \int|f^{+}-\gamma^{+}|d\mu<\epsilon/2, \int|f^{-}-\gamma^{-}|d\mu<\epsilon/2  $$
Then define $\gamma:=\gamma^{+}-\gamma^{-}$, so that
\begin{align*}
    \int|f-\gamma| d\mu&= \int|(f^{+}-\gamma^{+})-(f^{-}-\gamma^{-})|d\mu\\
    &\leq \int|f^{+}-\gamma^{+}|d\mu+\int|f^{-}-\gamma^{-})|d\mu\\
    & < \epsilon/2 + \epsilon/2 = \epsilon
\end{align*}
\subsection*{Exercise 6}
We know for fixed $\epsilon>0$, we can choose $N$ s.t. for $n\geq N$, $\sup_{x\in \Omega}|f_{n}(x)-f(x)|<\epsilon/\mu(\Omega)$. So then
$$\int f_{n}d\mu\leq \int(f+\epsilon/\mu(\Omega))d\mu, \int fd\mu\leq \int(f_{n}+\epsilon/\mu(\Omega))d\mu $$
then since we have that $\int\epsilon/\mu(\Omega)d\mu=\epsilon$,
$$\int f_{n}d\mu-\int fd\mu\leq \epsilon, \int fd\mu-\int f_{n}d\mu\leq \epsilon$$
Thus we get
$$ \int |f_{n}- f|d\mu\leq \epsilon$$
Hence
$$\int fd\mu=\lim_{n}\int f_{n}d\mu$$
\textit{In the case when $\mu(\Omega)=\infty$ we cannot take our $\epsilon$ over $\mu(\Omega)$, so the equality may fail.}
\end{document}
