\documentclass[twocolumn, 10pt]{article}
\usepackage{amsmath}
\usepackage{amssymb}

\author{jdRanda}
\title{Lecture 1}

\begin{document}
    \section{Complex Numbers}
        Defined as $ \mathbb{C} = \{ (x,y):x,y \in \mathbb{R} \} $
        subject to conditions,  for $(x_{1}, y_{1}), (x_{2}, y_{2}) \in \mathbb{C}$
        \begin{itemize}
            \item Addition ($+$):\\
            $ (x_{1}, y_{1}) + (x_{2}, y_{2}) = (x_{1}+x_{2}, y_{1}+y_{2})$
            \item Multiplication ($\cdot$):\\ $ (x_{1}, y_{1}) + (x_{2}, y_{2}) = (x_{1}x_{2}-y_{1}y_{2}, x_{2}y_{1}+x_{1}y_{2}) $
        \end{itemize}
        $(\mathbb{C},+), (\mathbb{C},\cdot)  $ are albelian groups, with units $(0,0)$ and $(1,0)$ respectivly
        \subsection{Lemma}
            $(\mathbb{C},+,\cdot) $ is a field with multiplicative inverse
            $$z \in \mathbb{C} \text{\textbackslash}  \{0,0\}, \quad
            z^{-1}=(\frac{x}{x^{2}+y^{2}}},{\frac{-y}{x^{2}+Y^{2}})$$\\
            such that $z*z^{-1}=(1,0)$\\\\

        \noindent We will define as follows:
        $$ 1:=(1,0),\quad i:=(0,1), 0:=(0,0) $$
        Allowing us to write complex numbers as

\end{document}
