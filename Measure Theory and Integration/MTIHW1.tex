\documentclass[a4paper, 12pt, twoside]{article}
\usepackage[left = 3cm, right = 3cm]{geometry}
\usepackage[english]{babel}
\usepackage[utf8]{inputenc}
\usepackage{amssymb}
\usepackage{amsmath}
\usepackage{amsthm}
\usepackage{multicol}

\author{jdRanda}

\begin{document}
\subsection*{Exercise 1}
\begin{itemize}
    \item[i)]

    \item[i)]
\end{itemize}
\subsection*{Exercise 6}
\subsection*{Exercise 10}
\begin{itemize}
    \item[i)]Want to show $f^{-1}(\emptyset)=\emptyset$\\
    Suppose to the contrary that $\exists x\in f^{-1}(\emptyset)$. This would imply $f(x)\in\emptyset$ which is a contradiction. Hence $f^{-1}(\emptyset)$ has no elements and thus $f^{-1}(\emptyset)=\emptyset$.
    \item[ii)]
    \item[iii)]
    \item[iv)]
    \begin{align*}
            x\in f^{-1}(\cup_{\alpha}E_{\alpha}) &\iff f(x)\in\cup_{\alpha}E_{\alpha}\\
            &\iff f(x)\in E_{\alpha_{1}} \quad\text{for some } \alpha_{1}\\
            &\iff x\in f^{-1}(E_{\alpha_{1}})\\
            &\iff x\in\cup_{\alpha}f^{-1}(E_{\alpha})
    \end{align*}
    \item[v)]
    \begin{align*}
            x\in f^{-1}(\cap_{\alpha}E_{\alpha}) &\iff f(x)\in\cap_{\alpha}E_{\alpha}\\
            &\iff f(x)\in E_{\alpha} \quad\forall\alpha\\
            &\iff x\in f^{-1}(E_{\alpha})\\
            &\iff x\in\cap_{\alpha}f^{-1}(E_{\alpha})
    \end{align*}
\end{itemize}



\end{document}
