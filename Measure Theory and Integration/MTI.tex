\documentclass[a4paper, 12pt, twoside]{article}
\usepackage[left = 3cm, right = 3cm]{geometry}
\usepackage[english]{babel}
\usepackage[utf8]{inputenc}
\usepackage{amssymb}
\usepackage{amsmath}
\usepackage{amsthm}
\usepackage{multicol}
% thanks tyler

\author{jdRanda}
\title{Measure Theory and Integration}

\begin{document}
\maketitle
\newpage

\section{The Riemann Integral}
\subsection{Definition}
Let $f:[a,b] \to  \mathbb{R}$ s.t. $a<b$ and let $P=\{x_{0},\dots,x_{n}\}$ be a partition of $[a,b]$ with $a=x_{0}<x_{1}<\dots<x_{n}=b$. Set
$$ L(P,f)= \sum_{i=1}^{n}\inf \{f(x):x_{i-1}\leq x < x_{i}\}(x_{i}-x_{i-1})$$
$$ U(P,f)= \sum_{i=1}^{n}\sup \{f(x):x_{i-1}\leq x < x_{i}\}(x_{i}-x_{i-1})$$
as the lower and upper integral sums respectivly. \\
Then f is integrable iff
$$ \sup_{P}L(P,f) = \inf_{P}U(P,f) $$
where the supremum and infinum are taken over all possible partitions.\\
Hence the Rieman Integral is
$$ \int_{a}^{b}f(x)dx=\sup_{P}L(P,f) = \inf_{P}U(P,f)$$
\subsection{Lemma}
Let $a,b\in \mathbb{R}$ s.t. $a<b$, $f:[a,b]\to \mathbb{R}$ a function. Then
$$ \inf_{P}U(P,f) \geq \sup_{P}L(P,f) $$
\begin{proof}
    Let $P_{1},P_{2}$ be partitions, define $\mathbb{Q}=P_{1} \cap P_{2}$. Since $Q$ is a 'finer' partition than both $P_{1}$ and $P_{2}$, it is clear
    $$ L(P_{1},f) \leq L(Q,f) \text{and } U(P_{2},f) \geq U(Q,f)$$
    Therefore
    $$ L(P_{1},f) \leq L(Q,f) \leq U(Q,f) \leq U(P_{2},f) $$
    Now as $P_{1},P_{2}$ are arbitrary, the equality is true for all $P$, so we can take the supremum of the LHS over all partitions, and the infinum of the RHS over all partitions, leaving us with...
    $$ \inf_{P}U(P,f) \geq \sup_{P}L(P,f)$$
\end{proof}
\subsection{Theorem}
A function $f:[a,b] \to  \mathbb{R}$ is \emph{Riemann Integrable} $\iff \forall \epsilon>0, \exists P_{*}$ s.t.
$$ U(P_{*},f)-L(P_{*},f)<\epsilon$$
Where $P_{*}$ is any partition on $[a.b]$
\begin{proof}
    Assume $ \forall \epsilon>0, \exists P_{*} \text{ s.t. }  U(P_{*},f)-L(P_{*},f)<\epsilon$, wts $f$ is Riemann Integrable.\\
    It is clear
    $$ \inf_{P}U(P,f)\leq U(P_{*},f) \text{ and } \sup_{P}L(P,f)\geq L(P_{*},f)$$
    Now by subtracting and using lemma 1.2 we obtain
    $$ 0\leq  \inf_{P}U(P,f)-\sup_{P}L(P,f) \leq U(P_{*},f)- L(P_{*},f) < \epsilon$$
    Since $\epsilon$ arbitrary take $\epsilon\to\infty$ and thus
    $$ \sup_{P}L(P,f) = \inf_{P}U(P,f) $$
    which is the definition of Riemann Integrability\\\\
    Now Assume $f$ is Riemann Integrable,\\
    Let $\epsilon>0$, then $\exists P_{1},P_{2}$ s.t
    $$U(P_{1},f) < \inf_{P}U(P,f) + \frac{\epsilon}{2} \text{ and } L(P_{2},f) > \sup_{P}L(P,f) + \frac{\epsilon}{2}$$
    Define $P_{*}=P_{1}\cup P_{2}$ and obtain
    \begin{align*}
         U(P_{*},f)-L(P_{*},f) &\leq U(P_{1},f) - L(P_{2},f)\\
        &< \inf_{P}U(P,f)-\sup_{P}L(P,f)+\epsilon \\
        &= \epsilon
    \end{align*}
\end{proof}
\subsection{Theorem}
$f:[a,b] \to  \mathbb{R}$ is Riemann Integrable iff the set of discontinuities of $f$ has \emph{Lebasque Measure Zero}.
\subsection{Definition}
The Lebasque Measure of an open interval $I=(a,b)$ is $\mu(I)=b-a$.\\
A set $N\subset \mathbb{R}$ has Lebasque Measure Zero if $\forall \epsilon > 0$ there exists a countable collection of open intervals $\{I_{1},I_{2},\dots\}$ s.t. $ N\subset \cup_{i=1}^{\infty}I_{i}$ and $\sum_{i=1}^{\infty}\mu(I_{i})<\epsilon$
\subsubsection{Examples}
Some examples of sets with a Lebasque Measure of Zero:
\begin{itemize}
    \item[-] A single point $\{a\}$
    \item[-] Any countable set of points $E=\{a_{1},\dots\}$
    \item[-] Any countable union of sets of measure $0$
    \item[-] Any subset of a set of measure $0$
\end{itemize}

\newpage
\section{Measurable sets and Integrals}
\subsection{Definition}
Let $X\neq\emptyset$ be a set. A family  $\mathbb{X}$ of subsets $X$ is a $\sigma$-algebra if
\begin{itemize}
    \item[i)] $\emptyset \in  \mathbb{X}$, $X \in \mathbb{X}$
    \item[ii)] $A\in \mathbb{X} \rightarrow A^{c} \in \mathbb{X}$
    \item[iii)] $A_{1},A_{2},\dots \in \mathbb{X} \rightarrow \cup_{n=1}^{\infty}A_{n}\in \mathbb{X}$
\end{itemize}
Now $(X,\mathbb{X})$ is called a \emph{Measurable Space}, and each $S\in\mathbb{X}$ are called \emph{Measurable Sets}.
\subsection{Definition}
Let $X\neq\emptyset$ and $\mathbb{A}$ be a non-empty collections of subsets of $X$. Let $\mathbb{Y}$ be the collection of all $\sigma$-algebras containing $\mathbb{A}$. Then $\beta(\mathbb{A})=\cap_{\mathbb{X}\in\mathbb{Y}}\mathbb{X}$ is the $\sigma$-algebra generated by $\mathbb{A}$. This is the smallest $\sigma$-algebra generated by $\mathbb{A}$.
\subsection{Definition}
Let $X=\mathbb{R}$, $\mathbb{A}=\{(a,b):a,b\in\overline{\mathbb{R}},a<b\}$. The $\sigma$-algebras generated by $\mathbb{A}$ is \emph{Borel Algebra}, denoted $\mathbb{B}$. This is the smallest $\sigma$-algebras contianing all open sets. A set $B\in \mathbb{B}$ is a \emph{Borel Set}.
\textit{NB: $\overline{\mathbb{R}}=\mathbb{R}\cup \{\pm\infty\}$}
\subsection{Definition}
Let $(X,\mathbb{X})$ be a measurable space. Then $f:X \to  \mathbb{R}$ is a $\mathbb{X}$-measurable function if for any Borel set $A\in \mathbb{B}$ we have $f^{-1}(A)\in \mathbb{X}$.
\subsection{Definition}
Let $(X,\mathbb{X})$, $(Y,\mathbb{Y})$ be measurable spaces. Then $f:X \to  Y$ is an $\mathbb{X}$-measurable function if for any set $A\in \mathbb{Y}$ we have $f^{-1}(A)\in \mathbb{X}$.
\newpage
\subsection{Lemma}
A function $f:X \to  \mathbb{R}$ is measurable $ \iff \forall \alpha \in \mathbb{R}$ the set
$$ \{x\in X:f(x)<\alpha\} \equiv f^{-1}((-\infty,\alpha))$$
is measurable (i.e belongs to $\mathbb{X}$).\\
\noindent
\underline{Proof}\\
Assuming the set is measurable, as $(-\infty,\alpha)\in \mathbb{B}$, then by definition $f^{-1}((-\infty,\alpha))\in \mathbb{X}$\\\\
Now assume $f^{-1}((-\infty,\alpha))\in \mathbb{X} \forall \alpha\in \mathbb{R}$, wts  for any Borel set $A\in \mathbb{B}$ we have $f^{-1}(A)\in \mathbb{X}$\\
Define $\mathbb{A}$ to be a collection of sets of the form $(-\infty,\alpha) $ with $\alpha \in \mathbb{R}$. Now we can see that sets in $\mathbb{A}$ generate Borel $\sigma$-algebra $\mathbb{B}$.\\
Let $\mathbb{Y}$ be the smallests $\sigma$-algebra contianing $\mathbb{A}$. Obviously $\mathbb{A}\subset \mathbb{B}$ and therefore $\mathbb{Y}=\beta(\mathbb{A})\subset \mathbb{B}$.\\
Now wts $\mathbb{Y}$ contians intervals $(a,b)$ for $a<b\in \mathbb{R}$. Take the sets $(-\infty,a),(-\infty,b) \in \mathbb{Y}$. Then we have
$$ [a,b)=(-\infty,b)\cap (-\infty,a)^{c}\in \mathbb{Y} \forall a<b\in \mathbb{R}$$
$$(a,b)=\cup_{n=N}^{\infty}[a- \frac{1}{n},b] \text{ for large enough N}$$
and hence $(a,b)\in \mathbb{Y}$.\\
Since $\mathbb{B}$ is the smallest $\sigma$-algebra contianing all open intervals $(a,b)$ we have $\mathbb{Y}=\mathbb{B}$.\\
Now we define the smallest $\sigma$-algebra contianing sets $f^{-1}(\mathbb{A})$, i.e. $\beta(f^{-1}(\mathbb{A}))$. Since $f^{-1}(\mathbb{A})\subset \mathbb{X}$ we also have $\beta (f^{-1} (\mathbb{A})) \subset \mathbb{X}$.\\
But $\beta(f^{-1}(\mathbb{A})) = f^{-1}(\beta(\mathbb{A} )) = f^{-1}(\mathbb{B}) $ and so $ f^{-1}(\mathbb{B}) \subset \mathbb{X}$ \qed
\subsection{Lemma}
Let $(X,\mathbb{X})$ be a measurable space, let $f:X \to  \mathbb{R}$. Then the following are equivalent:
\begin{itemize}
    \item[i)] $\forall\alpha\in \mathbb{R}, A_{\alpha}=\{x\in X:f(x)>\alpha\}\in \mathbb{X}$
    \item[ii)] $\forall\alpha\in \mathbb{R}, B_{\alpha}=\{x\in X:f(x)\leq\alpha\}\in \mathbb{X}$
    \item[iii)] $\forall\alpha\in \mathbb{R}, C_{\alpha}=\{x\in X:f(x)\geq\alpha\}\in \mathbb{X}$
    \item[iv)] $\forall\alpha\in \mathbb{R}, D_{\alpha}=\{x\in X:f(x)<\alpha\}\in \mathbb{X}$
\end{itemize}
\subsection{Lemma}
Let $(X,\mathbb{X})$ be a measurable space, $f,g:X\to \mathbb{R}$ be measurable. Let $F:\mathbb{R}^{2}\to \mathbb{R}$ be continuous. Then $h:X\to \mathbb{R}; h(x)=F(f(x),g(x))$ is measurable.
\begin{proof}
    Fix $\alpha \in \mathbb{R}$ and we want to show:
    $$X_{\alpha}:=\{x:F(f(x),g(x))<\alpha\}=\{x:(f(x),g(x))\in F^{-1}((-\infty,\alpha)) \}$$
    is measurable.\\
    Let $A=F^{-1}((-\infty,\alpha))$ and note it is an open set in $\mathbb{R}^{2}$ as $F$ is continuous, thus it is a countable union of open rectangles, so it is sufficient to show for any set $(a,b)\times(c,d)$ we have that
    $$\{x:(f(x),g(x))\in(a,b)\times(c,d)\}=\{x:a<f(x)<b\}\cap\{x:c<g(x)<d\} $$
    is measurable. We can now use the fact $f,g$ are measurable.
\end{proof}
\subsection{Corollary}
Let $f,g$ be measurable and $c\in \mathbb{R}$. Then
$$f+g,fg,|f|,\frac{f}{g}\quad g\neq0,\max\{f,g\},\min\{f,g\},cf$$
are all measurable by above lemma.
\subsection{Definition}
Let $(X,\mathbb{X})$ measurable space. Then $f:X \to  \overline{\mathbb{R}}$ is $\mathbb{X}$-measurable if for any $\alpha\in \mathbb{R}$, the set $\{x\in X:f(x)>\alpha\}\in \mathbb{X}$.\\
The collection of all $\overline{\mathbb{R}}$-valued, $\mathbb{X}$-measurable functions on $X$ is denoted $M(X,\mathbb{X})$.
\subsection{Lemma}
Let $(X,\mathbb{X})$ be a measurable space. $f:X\to \overline{\mathbb{R}}$ is measurable iff:
\begin{itemize}
    \item[i)] $A=\{x\in X:f(x)=+\infty\}$\\
    $B=\{x\in X:f(x)=-\infty\}$
    \item[ii)] the function $f_{1}(x)=
    \begin{cases}
      f(x), & x\in(A\cup B)^{c} \\
      0, & x\in A\cup B
   \end{cases}$ is measurable.
\end{itemize}
\begin{proof}
    Suppose $f_{1},A,B$ are measurable. For $\alpha\geq0$ we have
    $$\{x:f(x)>\alpha\}=\{x:f_{1}(x)>\alpha\}\cup A $$
    And thus is measurable, analogous for $\alpha<0$.\\
    Now suppose f is measurable.
\end{proof}

\subsection*{Definition 2.14}
A \emph{simple function} is a finite linear combination of characteristic functions of measurable sets, i.e. $f:X\to \mathbb{R}$ is simple if $f(x)=\sum_{i=1}^{n}a_{i} \chi(A_{i})$, were $a_{i}\in \mathbb{R}, A_{i}\in \mathbb{X}$.
\subsection*{Lemma 2.15}
Let $f\in M(X,\mathbb{X}),f\geq0$. Then there exists a sequence $(\phi_{n})$ in $M(X,\mathbb{X})$ such that
\begin{itemize}
    \item[i)] $0\leq \phi_{n}(x)\leq\phi_{n+1}(x)\quad\forall x\in X, n\in \mathbb{N}$
    \item[ii)] $\lim_{n\to\infty}\phi_{n}(x)=f(x)$
    \item[iii)] Each $\phi_{n}$ is a smiple function.
\end{itemize}
\begin{proof}
Fix $n\in \mathbb{N}$ and for $k\in \{0,\dots,n2^{n}-1\}$ define
$$E_{k,n}=\{x:f(x)\in[\frac{k}{2^{n}},\frac{k+1}{2^{n}}] \text{ and } E_{n2^{n},n}= \{x:f(x)\geq n\}$$
We then define
$$\phi_{n}(x)= \frac{k}{2^{n}} \text{ if } x\in E_{k,n}$$
Giving us
$$ \phi_{n}=\sum_{k=0}^{2^{n}n} \frac{k}{2^{n}}\chi E_{k,n}$$
which is a simple funciton. \\\\
Here we are splitting the interval $[0,n)$ into $n2^{n}$ parts and approximating $f$ by its lowest value within the interval. If $f(x)\geq n$ we approximate $f(x)$ by $n$. Since $f$ is measurable we know sets $E_{k,n}$ and $E_{n2^{2},n}=\{x:f(x)\geq n\}$ are measurable, so $\phi_{n}$ is indeed simple.\\\\
Now we check pointwise convergence of $\phi_{n}$ to $f$. Let $x\in X$ and then either $f(x)=\infty, f(x)<\infty$.\\
Assume $f(x)=\infty$ then it is clear $\phi_{n}(x)=n$ and as $n\to\infty, \phi_{n}(x)\to f(x)$.\\
Assume $f(x)<\infty$ then $f(x)<N,  N\in \mathbb{N}$ and $x\in E_{k,n}$ for some $k\in \mathbb{N},n\geq N$. However in every set $E_{k,n}$ we have $|\phi_{n}(x)-f(x)|<\frac{1}{2^{n}}$. Taking $n\to\infty$ we have $\phi_{n}(x)\to f(x)$. Therefore pointwise convergence.\\\\
Now we check $\phi_{n}\leq \phi_{n+1}$.\\
Assume $x\in E_{(n+1)2^{n+1},n+1}$ then $\phi_{n+1}(x)=n+1>\phi_{n}(x)$.\\
Assume $x\in E_{k,n+1}$ for some $k\in\{0,\dots,(n+1)2^{n+1}-1$. Then $\phi_{n+1}(x)=\frac{k}{2^{n+1}}$. But we also have $x\in E_{[k/2],n}$ implying $\phi_{n}(x)=\frac{[k/2]}{n}$. Hence $\phi_{n+1}(x)\geq\phi_{n}(x)$.
\end{proof}
\subsection{Definition}
A function $f:X \to  \mathbb{R}$ is called \emph{elementary} if it is measurable and take no more than a countable number of values, i.e.
$$f(x)=\sum_{i=1}^{\infty}a_{i}\chi(A_{i}) $$
where $a_{i}\in \mathbb{R}, A_{i}=\{x\in X: f(x)=a_{i}\}$.
\subsection{Lemma}
A function $f:X \to  \mathbb{R}$ is measurable iff it is a limit of a uniformly convergent sequence of elementary functions.
\begin{proof}
    Let $\{f_{n}\}$ be a sequence of elementary function and $f_{n}\to f$ uniformly on $X$. Uniform convergence implies pointwise convergence so we are done using lemma 2.15.\\\\
    Now let f be a measurable function, define a sequence $\{f_{n}\}$ s.t.
    $$f_{n}(x)=\frac{m}{n} \text{ on } A_{n}^{m}=\{x\in X: \frac{m}{n}\leq f(x)< \frac{m+1}{n}\},m\in \mathbb{Z}, n\in \mathbb{N} $$
    Clearly $f_{n}(x)$ is an elementary function and $|f_{n}(x)-f(x)|\leq \frac{1}{n}$ on X.
\end{proof}
\subsection{Lemma}
Let $(X,\mathbb{X})$ be a measuable space, $f:X \to  \mathbb{R}$ be $\mathbb{X}$-measurable, and $\phi:\mathbb{R} \to  \mathbb{R}$ be $\mathbb{B}$-measurable. Then $g(x)=\phi(f(x))$ is $\mathbb{X}$-measurable.
\begin{proof}
    
\end{proof}







\end{document}
