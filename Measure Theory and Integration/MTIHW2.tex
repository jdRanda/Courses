\documentclass[a4paper, 12pt, twoside]{article}
\usepackage[left = 3cm, right = 3cm]{geometry}
\usepackage[english]{babel}
\usepackage[utf8]{inputenc}
\usepackage{amssymb}
\usepackage{amsmath}
\usepackage{amsthm}
\usepackage{multicol}

\author{Joseph}

\begin{document}
Joseph Dowling.
\subsection*{Exercise 1}
Want to show for fixed $A\in \mathbb{U},\forall E\in \mathbb{U}$, $\lambda(E)=\mu(A\cap E)$ is a measure.
\begin{itemize}
    \item[I)] $\lambda(\emptyset)=\mu(A\cap\emptyset)=\mu(\emptyset)=0$
    \item[II)] Let $E\in \mathbb{U}$. $\lambda(E)=\mu(E\cup A)\geq 0$ as $\mu$ is a measure and $E\cup A \subseteq A\in \mathbb{U} $.
    \item[III)] Let $E_{1},E_{2},\dots$ be disjoint subsets of $\mathbb{U}$ such that $\cup_{i}E_{i}=E\in \mathbb{U}$.
    \begin{align*}
        \lambda(\cup_{n=1}^{\infty}E_{n}) &= \lambda(E)\\
        & = \mu(A\cap E) = \mu(\cup_{n=1}^{\infty}(A\cap E_{n}))\\
        & = \sum_{n=1}^{\infty}\mu(A\cap E_{n})\\
        & = \sum_{n=1}^{\infty}\lambda(E_{n})
    \end{align*}
\end{itemize}
Hence we have that $\lambda$ is a measure on $\mathbb{U}$.
\subsection*{Exercise 8}
Let $X$ be a set of at least two elements. Set $\mathbb{X}=\{\emptyset,X\}$. Then trivially $(X,\mathbb{X})$ is a measure space. Define $\mu:\mathbb{X}\to\overline{\mathbb{R}}$ by $\mu(X)=\mu(\emptyset)=0$. Then $(X,\mathbb{X},\mu)$ is not complete as $\exists A\subset X$ with $A\not\in \mathbb{X}$ but $\mu(X)=0$.
\subsection*{Exercise 12}
\begin{itemize}
    \item[a)]
    Want to show $f_{n}\to f$ uniform.\\
    Let $\epsilon>0, N>\frac{1}{\epsilon}$. Then for $n\geq N$ there are two cases. Case 1 for $x\in[0,n]$\\
    $$|f_{n}(x)-f(x)|=|\frac{1}{n}-0|=\frac{1}{n}\leq\frac{1}{N}<\epsilon$$
    Case 2 for $x\not\in[0,n]$
    $$|f_{n}(x)-f(x)|=0<\epsilon $$
    Now want to show $\int fdm\neq \lim\int f_{n}dm$.\\
    We know trivially that $\int fdm=0$ as $f$ is the zero function. So Now
    $$\lim_{n}\int f_{n}dm=\lim_{n}\frac{1}{n}m([0,n])=\lim_{n}\frac{n}{n}=1 $$
    So we have that
    $$ \int fdm=0\neq 1=\lim_{n}\int f_{n}dm$$
    Fatous lemma is applicable here but, this does not contradict the Monotone Convergence Theorem as $(f_{n})$ is not monotone non-decreasing.
    \item[b)]
     Want to show $\int gdm\neq \lim\int g_{n}dm$.\\
     We know trivially that $\int gdm=0$ as $g$ is the zero function. So Now
     $$\lim_{n}\int g_{n}dm=\lim_{n}nm([\frac{1}{n},\frac{2}{n}])=n(\frac{2}{n}-\frac{1}{n})=1 $$
     So we have that
     $$ \int gdm=0\neq 1=\lim_{n}\int g_{n}dm$$
     $(g_{n})$ converges uniform to $g$, Fatous lemma is applicable here, and $(g_{n})$ is monotone increasing so MCT applies also.
\end{itemize}

\end{document}
