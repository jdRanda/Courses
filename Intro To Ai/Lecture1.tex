\documentclass[10pt]{article}

\author{jdRanda}
\title{Lecture 1}



\begin{document}
    \maketitle


    \section{Machine Learning}
        A program is said to learn from experiance E with respect to some task(s) T and
        performance measure P, if its performance at tasks in T, measured by P, improves
        with E.\\
        There are three main types:

        \begin{itemize}
            \item Supervised
            \item Unsupervised
            \item Reinforcement Learning
        \end{itemize}

    \section{Big Data}

        \subsection{Bonferroni's Principle}
        A risk when data-mining big data is that you will “discover” patterns that are meaningless. Known as Bonferroni’s principle:\\
        (roughly) if you look in more places for interesting patterns than your amount of data will support, you are bound to find crap.

\end{document}
